\documentclass[a4paper, 11pt]{article}
\usepackage[english]{babel}  
\usepackage[utf8]{inputenc} 
\usepackage[centertags]{amsmath}
\usepackage{amssymb,amsfonts,amsthm}
\usepackage{indentfirst}
\usepackage{array}
\usepackage{float}
\usepackage[usenames,dvipsnames,svgnames,table]{xcolor}
\usepackage{hyperref}
\usepackage{enumerate}
\usepackage{graphicx}
\usepackage{epstopdf}
\usepackage{graphics}
\usepackage{subfigure}
\usepackage[margin=1.5cm,footskip=1.5cm,includefoot]{geometry} 
\usepackage{listings}
\usepackage{setspace}
\usepackage{multirow}

\setstretch{1.1}

\begin{document}

	\begin{center}
		\sc\large\textbf{Personalized Apartment Recommender}\\
			\noindent {\small 
        	Ansel Lim,
        	Daosheng Lin
        	Hong Wen Tai,
        	Keith Loo (Team 153)}\\
        	
	\end{center}
	
	
All team members contributed equally to the project.  Ansel performed data collection \& scoring. Scoring algorithm was designed by Key. Daosheng designed the user preferences questionnaire. Keith created the map.

	\section{Introduction}
	
	Aspiring homeowners in Singapore are constrained by biases in multiple information sources, as the content creators on many platforms tend to be realtors, who have skin in the game. We have created a tool which aspiring homebuyers in Singapore can use to analyze and visualize multiple unbiased sources of information, and provide recommendations to the users based on their specified preferences and constraints. 

	\section{Problem Definition}
	
	The continuously rising property prices in Singapore are making homes less affordable for aspiring homeowners. Our project aims to help buyers find affordable housing based on their preferences and financial constraints. B.H. Chua (2015) [1] and C. Y. Yiu \& C. M. Hui (2005) [2] discussed how use of buyers’ retirement savings in the Central Provident Fund to finance home ownership has resulted in persistent increases in property prices. Both articles suggest that property prices in Singapore will have an increasing trend but did not discuss factors that impact the movement of property prices. Teck Chin \& Geofrey Mills (1999) [3] discusses the importance of timing strategies for loss avoidance, highlighting the impact of property transaction expenses and financial strength of property investors. Home buyers in Singapore typically obtain their information through (a) social media (e.g. blogs, Instagram, YouTube, forums), (b) browsing commercial real estate websites (e.g. PropertyGuru, 99.co), (c) online advertisements, (d) messaging applications (e.g. WhatsApp, Telegram), (e) old media (e.g. newspapers, television) and (f) word of mouth. Multiple sources of information may lead to confusing and conflicting conclusions. Furthermore, as alluded to by Kavey (1989) [4], the search for homes may be constrained by the bias present in the information provided by realtors.

	\section{Methodology}
	
	\subsection{Intuition}
	
	Most of the currently available tools for researching the property market involve filtering among certain attributes, such as location, price and size of the home. However, the thought process in purchasing a home involves a much greater variety of factors, such as proximity to various amenities (e.g. food \& beverage, schools, recreational facilities). Each individual would have his/her own preferences and priorities, hence it is difficult to have a one-size-fits-all approach in determining the best place to live. We have built a tool to make this a simple yet dynamic process, where the recommender system will take into account user preferences and return a list of best options.
	
	\subsection{Data collection and preprocessing}
	
	The Government of Singapore maintains and publishes publicly available datasets (www.data.gov.sg and www.ura.gov.sg/maps/api/), and a comprehensive map with an open API (www.onemap.gov.sg), both of which are extremely useful for our tool. Through these sources, we were able to download and scrape details of residential projects (e.g. past sale transactions, addresses, map coordinates) and locations of various amenities (e.g. parks, educational institutions, food and beverage establishments). We further enriched the information with data from Google Places API, adding details such as review ratings and number of reviews. The raw data was stored in multiple comma separated values (CSV) files, and subsequently formatted into a consistent schema and loaded into an SQLite database. The data was stored in two tables, (1) a “Properties” table, containing details of the various residential housing projects such as name, district, location, average transacted price, and (2) a “Features” table, containing details of the different amenities such as name, category, location, and review ratings. In total, we retrieved 10,614 rows of property transaction data and 32,695 rows of amenities data over 15 different types of amenities (

    \subsection{User preferences questionnaire}
    
    The landing page of our tool is a user preferences questionnaire, where the user answers a set of questions for us to determine the limitations and preferences specific to this user. This consists of a mix of “fixed constraints”, where homes falling out of this domain would not be recommended (e.g. if the user indicates price per square metre below S\$10,000, no homes with average transacted price above this value will be shown), and “preferences”, where the user can indicate how important each attribute should be (e.g. a young parent may strongly prefer schools and parks, but be less inclined towards gyms). We developed the interactive questionnaire using a combination of Javascript and Python, with Flask as the web framework. After answering the required questions, the resulting values of the constraints and preferences are stored and handed over to the scoring system.
    
    [ PUT IMAGE HERE ]
    
    \subsection{Review of scoring methodology in recommendation systems} 
    
    The application calculates scores for different housing projects, and presents the top-scoring options to users.
    
    Various insightful methods in decision science exist in the literature. B. Song \& S. Kang (2015) [5] propose assignment of “weights” to different user inputs and attributes according to user-specified rankings, which we may adopt. However, this is not a dynamic solution; the process of arriving at the calculated weights is not ideal in a web application as it is difficult to keep users engaged during the intervening computation time. M. Morita \& Y. Shinoda (1994) [6] explore user reading times as a gauge of the relevancy of search results. This concept of using user engagement attributes may be explored in the design of our scoring algorithm. S. Opricovic \& G.H. Tzeng (2002) [7] discuss the use of two multiple-criteria decision-making methods, VIKOR and TOPSIS, in finding optimal solutions, however this may not be computationally feasible in large-scale evaluation of hundreds or thousands of alternative results. 
    
    Isinkaye, F. O., Folajimi, Y. O., \& Ojokoh, B. A. (2015) [8] discuss different techniques used in real estate recommender systems such as content-based and collaborative filtering methods. Deldjoo, Y., Dacrema, M.F., Constantin, M.G., et al. (2019) [9] explore a novel hybrid method which not only performs better than a pure content-based filtering method, but is also able to deal with the “cold start problem”. However, the paper had existing user-item interaction data which is not available to us from publicly available data sources. Therefore, we will have to use a pure content-based approach that relies on item features. 
    
    Interestingly, Lops, P., Jannach, D., Musto, C. et al. (2019) [10] share about developments for content-based recommendations such as enriching features using multimedia data or using embeddings to encode item features. We can possibly explore using such methods to improve our recommendations. 
    
    Computing similarity between houses can be a very computationally intensive process. J. Leskovec, A. Rajaraman, \& J. D. Ullman (2014) [11] discuss several methods of finding similar items efficiently, such as using cosine distance to compare similarity between items and using locality-sensitive hashing to hash similar items into buckets. This effectively reduces the search space to items which have the same bucket. By confining our search space to houses within the same hash bucket, we may recommend homes more efficiently. 

    \subsection{Our scoring methodology}
    
    Putting our research into application, our scoring method takes into account input weights from the user, as well as attribute scores for each housing project. Both the input weights and attribute scores have a range of 0 to 1, and the overall score will be the weighted sum of scores. Assuming there are 2 attributes (schools and public transport). The user places less importance on schools and gives a weight of  0.3 (wschool = 0.3), and values the public transport links in the area thus assigns a weight of 0.8 to public transport (wtransport = 0.8). Each housing project has a pre-computed score for each attribute; housing project A can have 0.8 as its score for schools (sA,school = 0.8) and 0.5 as its score for public transport (sA,transport = 0.5), whereas housing project B can have 0.4 as its score for schools (sB,school = 0.4) and 0.9 as its score for public transport (sB,transport = 0.9). Using the above examples, the overall score of housing project A will be:
    
\[ w_{\text{school}} \times s_{A,\text{school}} + w_{\text{transport}} \times s_{A,\text{transport}} = 0.3 \times 0.8 + 0.8 \times 0.5 = 0.64 \]

Similarly, the overall score for housing project 2 will be:

    \[ w_{\text{school}} \times s_{B,\text{school}} + w_{\text{transport}} \times s_{B,\text{transport}} = 0.3 \times 0.4 + 0.8 \times 0.9 = 0.84 \]

To derive scores for individual attributes, we have to consider the type of attribute. We computed up to 2 types of scores per attribute: a quantity score and quality score. For attributes where the ratings are important (e.g. certain schools may be perceived to be better than others), both quality and quantity scores will be calculated. Other attributes where the concept of ratings may not be as relevant (e.g. Subway station) will only have a quantity score assigned. Each type of score will also have its own weight, so the user will input weight for school quantity score, as well as school quality score. Therefore, the overall score for a housing project will be:

\[ w_{\text{school quality}} \times s_{B,\text{school quality}} + w_{\text{school quality}} \times s_{B,\text{school quality}} + w_{\text{transport}} \times s_{B,\text{transport}} \]

For the computation of the quantity score, we computed the number of nearby schools for a given housing project. We then plotted a histogram to analyze the distribution of schools across various properties in Singapore. Depending on the nature of the specific attributes (in this case schools) and the shape of the distribution, we assigned a score, applying a higher score for housing projects with more schools in the vicinity. Thereafter, we applied scaling to keep the score range between 0 and 1. For the quality score, we first computed the median rating across all nearby schools for a given housing project, followed by min-max scaling so that the range of the score will be kept between 0 and 1. We chose to take the median instead of average, since it will be more representative and avoid situations where quality score is largely affected by one anomalous poor school. This process was repeated for all attributes.

 As our ratings are scraped from user reviews, we have to consider the number of reviews in conjunction with the average ratings. If a school rated 5 stars has only 1 review, it may not be necessarily better than another school rated 4.5 stars with 1,000 reviews. To take into account both the rating and number of reviews, we shall use a weighted average of individual and global ratings: 
\[ R_a = W\times R +(1-W)\times R_0\] where $R_a$ is the weighted rating, $W$ is the weight factor (number of reviews for a particular school / maximum reviews for all schools), $R$ is the rating for the school, and $R_0$ is the global average rating (average of all schools in our dataset). Using this formula, if a school has only a low number of reviews, W will be small, and consequently, the output rating Ra will take a higher weight from the global average rating instead of the individual school rating. This approach gives us an accurate ranking of schools, taking into account rating and number of votes.

\subsection{Database design \& data retrieval}

The pre-computed scores for each attribute and housing unit are stored in a table in an SQLite3 database. After the user has indicated his/her preferences for the various attributes, the following workflow will be triggered:
\begin{enumerate}
    \item The table will be queried from the SQLite3 database.
    \item Based on the indicated preferences for the fixed constraints (eg. district, price per square metre), the table will be filtered accordingly to contain suitable housing units.
    \item An overall score will be computed for each housing project, using the user-input weights for each attribute as well as the attribute’s score.
    \item The list of housing projects will then be sorted by the overall score in descending order, and the top results will be recommended to the user.
\end{enumerate}

\subsection{Data visualization}

Upon deriving the recommended results for users, the next step would be to show the recommendations to our users. We utilised the Singapore Government’s OneMap interface (www.onemap.gov.sg) for this step. The scoring system returns the top 5 results in terms of overall scores, along with JSON objects containing details of the various amenities within a 1-kilometer radius of these residential projects. These results are mapped out in the interface in the form of pins. Clicking on a pin allows the user to drill-down into greater detail. Upon drilling in, the user has the option of showing or hiding “layers”, each of which would contain the specific amenities within the 1-kilometer radius. Map data and visualizations have been used in real estate valuation, such as in Yang, J. et al. (2021) [12], where the authors applied machine learning techniques to urban greenery data on Google Maps. In line with home buyers’ increasing environmental consciousness, we included the visualization of public parks within our application. 

[ SHOW IMAGES HERE ]

\section{Evaluation}

We conducted experiments to evaluate the effectiveness of our tool.

\subsection{Experimental methodology}

As our tool is meant to recommend homes according to the user-defined preferences, the key question is simply “are the recommended homes suitable for the user?”. While this sounds simple, each user will have a different set of constraints and preferences, thus the answer will differ for each individual. We tested the effectiveness of our tool by creating several “personas” of users and their assumed preferences, and observed the results recommended by our tool. The details of our testbed are elaborated below.

\subsection{Experimental results}

The following user personas, along with assumptions, have been run against our tool. For each persona, we expected that the results returned will only include those meeting the fixed constraints defined, and the preferred amenities will rank among the top quartile of homes across Singapore. For each persona, we tested the results to ensure that the scoring system is producing accurate and relevant recommendations. For the purposes of the experiment, we gave the maximum weightage (100) for preferred features, while leaving the rest of the weights at the lower value of 20. We also ran experiments for the data retrieval process, to evaluate the time taken to compute the results and make recommendations:


\begin{table}[ht]
    \centering
    \begin{tabular}{p{0.2\linewidth} | p{0.3\linewidth} | p{0.4\linewidth}} \hline
      & \textbf{Input parameters}  & \textbf{Results} \\ \hline
      \textbf{Fixed constraints }& Price limit of \$5,000 per sq m
Districts: D27, D28  & As expected, only results meeting these criteria are shown. \\ \hline
\textbf{Preferred features} & Schools, Clinics, Hawker centres, MRT stations, Bus stops. & The top 5 results returned highest possible scores for the quantity score of clinics, hawker centres,  primary schools and secondary schools based on the available projects within the fixed constraints. The other preferred features returned better than average scores, but were not among the top quartile. \\\hline
    \end{tabular}
    \caption{Persona 1: young parents with below-average income}
    \label{tab:my_label}
\end{table}

\begin{table}[ht]
    \centering
    \begin{tabular}{p{0.2\linewidth} | p{0.3\linewidth} | p{0.4\linewidth}} \hline
      & \textbf{Input parameters}  & \textbf{Results} \\ \hline
      \textbf{Fixed constraints }& Price limit of \$8,000 per sq m
Districts: All & As expected, only results meeting these criteria are shown. \\ \hline
\textbf{Preferred features} & 
Sports facilities, Gyms, Community centres, Parks, Hawker centres, Eating establishments, MRT stations, Bus stops, Taxi stands. & 
The top 2 results returned highest possible scores for quantity and quality of all the preferred features. Results 3 to 5 showed slightly lower scores for community centres and gyms. 

 \\\hline
    \end{tabular}
    \caption{Persona 2: single young adult with average income}
    \label{tab:my_label}
\end{table}


\begin{table}[ht]
    \centering
    \begin{tabular}{p{0.2\linewidth} | p{0.3\linewidth} | p{0.4\linewidth}} \hline
      & \textbf{Input parameters}  & \textbf{Results} \\ \hline
      \textbf{Fixed constraints }& Price range between \$20,000 and \$30,000 per sq m. Districts: D9, D10, D11 
 & As expected, only results meeting these criteria are shown. \\ \hline
\textbf{Preferred features} & 
Schools, Clinics, Sports facilities, Gyms, Community centres, Parks, Malls. & 
\textbf{TO FILL UP}

 \\\hline
    \end{tabular}
    \caption{Persona 3: large multi-generational family, wealthy.
}
    \label{tab:my_label}
\end{table}

For all of the 3 personas tested above, the tool took under 2 seconds between the questionnaire submission to the display of the top 5 results, which is excellent from the perspective of user experience.

\section{Discussion}

In the first iteration of the tool, we had calculated the quantity scores of the various attributes based on 25-50-75th percentile. However, the results were not ideal, as the distribution of certain attributes (e.g. number of malls in the area) tended to fall within a fairly small range. As a result, the “top quartile” thresholds tended to be close to the median value. Instead of using the 25-50-75th percentile method, we changed our methodology by plotting out the distributions of all 15 attributes one by one, and assigned bins of non-uniform widths to each attribute. For instance, the number of primary schools within the vicinity of a home was largely 2, causing the 25th and 50th percentile thresholds to both fall at 2. Applying our new methodology, having 1 school in the vicinity would result in a score of 0.25, 2 schools will score 0.5, 3 schools will score 0.75, and 4+ schools will score 1.0. With this change, the top results tended to stand out more than the average housing project.

[ SHOW IMAGE ]

Similarly, for quality scores, we initially calculated them based on 25-50-75th percentile. However, since the median weighted ratings of amenities for a given housing project tended to be very close to one another, this resulted in very similar values in each of the bins. To remedy this, we applied min-max scaling instead, which allowed housing projects with high median weighted ratings of amenities to stand out more.

We are satisfied with the performance of the tool, as it helps to identify housing projects in ideal areas in an objective way and the interface is simple and intuitive. Even if the housing projects do not meet the user’s specific requirements, the suggestions indicate the appeal of the suggested areas and zones, and the user can thus focus their search on properties in these areas. 

The performance of the tool is not without limitations. We noted that some of the results were located in the same vicinity, within a close proximity to one another. This is expected, since these housing projects would have similar features and scores. Further development can be done to enhance the tool in the future, perhaps by plotting out a heatmap of overall scores, instead of simply showing the top 5 results. Furthermore, the freely available data was limited to a certain extent, with no information about the age of the housing projects. As older homes tend to fetch a lower price as compared to new ones, setting an upper price limit as a constraint could lead to a bias toward these older homes.  





	\begin{thebibliography}{00} \footnotesize
		\bibitem{BengHuatChua}
		Beng Huat Chua (2015). \textit{Financialising Public Housing as an Asset for Retirement in Singapore}, International Journal of Housing Policy, 15:1, 27-42.
		\bibitem{CYYiu} C. Y. Yiu\& C. M. Hui (2005). \textit{An Empirical Study of the Impact of Income Uncertainty on Private Residential Property Markets in Singapore and Hong Kong}, Housing Studies, 20:5, 753-769.
		\bibitem{TeckChin} Teck Chin\& Geofrey Mills (1999). \textit{An Optimal Control Approach to Market Timing in the Singapore Property Market}, Journal of Real Estate Portfolio Management, 5:1, 83-94.
		\bibitem{Kavey} Kavey, Lee (1989). \textit{Condominium Search and Purchase Process}. Research Thesis. National University of Singapore, Singapore. Accessed at scholarbank.nus.edu.sg/handle/10635/163970.
		\bibitem{BangweonSong} Bangweon Song\& Seokjoong Kang (2015). \textit{A Method of Assigning Weights Using a Ranking and Nonhierarchy Comparison}, Advances in Decision Sciences, 2090-3359
		\bibitem{Morita} Morita, Masahiro\& Shinoda, Yoichi. (1994). \textit{Information Filtering Based on User Behavior Analysis and Best Match Text Retrieval}. 272-281. 10.1007/978-1-4471-2099-5\_28.
		\bibitem{Opricovic} Opricovic, Serafim\& Tzeng, Gwo-Hshiung. (2004). \textit{Compromise solution by MCDM methods: a comparative analysis of VIKOR and TOPSIS}, European Journal of Operational Research. 156. 445-455. 10.1016/S0377-2217(03)00020-1. 
		\bibitem{Isinkaye} Isinkaye, F. O., Folajimi, Y. O.,\& Ojokoh, B. A. (2015). \textit{Recommendation systems: Principles, methods and evaluation}, Egyptian Informatics Journal, 16(3), 261-273.
		\bibitem{Deldjoo} Deldjoo, Y., Dacrema, M.F., Constantin, M.G. et al. (2019). \textit{Movie genome: alleviating new item cold start in movie recommendation}, User Model User-Adap Inter 29, 291-343.
		\bibitem{Lops} Lops, P., Jannach, D., Musto, C. et al. (2019). \textit{Trends in content-based recommendation}, User Model User-Adap Inter 29, 239-249.
		\bibitem{Leskovec} Jure Leskovec, Anand Rajaraman, and Jeffrey David Ullman. (2014). \textit{Mining of Massive Datasets (2nd. ed.)}, Cambridge University Press, USA.

	\end{thebibliography}
\end{document}